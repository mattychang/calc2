\documentclass{article}

\usepackage{amsmath, amssymb, sectsty, graphicx, array}
\DeclareMathOperator{\sech}{sech}
\DeclareMathOperator{\arcsec}{arcsec}
\DeclareMathOperator{\arcsinh}{arcsinh}
\DeclareMathOperator{\arccosh}{arccosh}
\DeclareMathOperator{\arctanh}{arctanh}
\subsectionfont{\normalfont\large}
\subsubsectionfont{\normalfont\large}

\usepackage{tocloft}
\setcounter{tocdepth}{1}

\title{MATH 122: Quiz Review Solutions}
\author{written by someone who has too much free time}

\begin{document}
\maketitle
\tableofcontents
\newpage

\section{Quiz 2 Review}
\subsection{
	\begin{align*}
		\int{x\sin{(x^2 + 5)}dx}
	\end{align*}
}
\subsection{
	\begin{align*}
		\int{\sin^4{x}\cos{x}dx}
	\end{align*}
}
\subsection{
	\begin{align*}
		\int{\frac{1}{x\ln{x}}dx}
	\end{align*}
}
\subsection{
	\begin{align*}
		\int{x^2 \ln{x}dx}
	\end{align*}
}
\subsection{
	\begin{align*}
		\int{x^2 \cos{x} dx}
	\end{align*}
}
\subsection{
	\begin{align*}
		\int{x^2 e^{2x}dx}
	\end{align*}
}
\subsection{
	\begin{align*}
		\int{xe^{x^2}dx}
	\end{align*}
}
\subsection{
	\begin{align*}
		\int{\cos^3{2x}dx}
	\end{align*}
}
\subsection{
	\begin{align*}
		\int{\cos^2{3x} \sin^2{3x}dx}
	\end{align*}
}
\subsection{
	\begin{align*}
		\int{\sin^5{x} \cos^2{x} dx}
	\end{align*}
}
\subsection{
	\begin{align*}
		\int{\sec^3{x} \tan^3{x} dx}
	\end{align*}
}
\subsection{
	\begin{align*}
		\int{\sec^4{x} \tan^2{x} dx}
	\end{align*}
}
\subsection{
	\begin{align*}
		\int{\sqrt{\tan{x}} \sec^4{x} dx}
	\end{align*}
}
\subsection{
	\begin{align*}
		\int{\sqrt{\sec{x}} \tan^3{x} dx}
	\end{align*}
}


\newpage
\section{Quiz 3 Review}
%2.1
\subsection{
	\begin{align*}
		\int{\frac{x^3}{\sqrt{1 - x^2}} dx}
	\end{align*}
}

%2.2
\subsection{
	\begin{align*}
		\int{\frac{1}{\sqrt{x^2 + 4}} dx}
	\end{align*}
}

%2.3
\subsection{
	\begin{align*}
		\int{\frac{1}{(16 - x^2)^{3/2}} dx}
	\end{align*}
}

%2.4
\subsection{
	\begin{align*}
		\int{\frac{1}{\sqrt{4x^2 + 9}}dx}
	\end{align*}
}

%2.5
\subsection{
	\begin{align*}
		\int{\tanh{x} dx}
	\end{align*}
}

%2.6
\subsection{
	\begin{align*}
		\int{\frac{\cosh{x}}{3\sinh{x} + 4} dx}
	\end{align*}
}

%2.7
\subsection{
	\begin{align*}
		\int{\arctanh{x} dx}
	\end{align*}
}

%2.8
\subsection{
	\begin{align*}
		\int{\frac{2 - x}{x^2 + x} dx}
	\end{align*}
}

%2.9
\subsection{
	\begin{align*}
		\int{\frac{3x + 11}{x^2 + 5x + 6} dx}
	\end{align*}
}

%2.10
\subsection{
	\begin{align*}
		\int{\frac{x^3 + 6x^2 + 3x + 6}{x^3 + 2x^2} dx}
	\end{align*}
}

%2.11
\subsection{
	\begin{align*}
		\int{\frac{2x^2 + 5x - 1}{x^3 + x^2 - 2x} dx}
	\end{align*}
}

%2.12
\subsection{
	\begin{align*}
		\int{\frac{3x + 6}{x^3 + 2x^2 - 3x} dx}
	\end{align*}
}

%2.13
\subsection{
	\begin{align*}
		\int{\frac{6x^2 - 10x + 2}{x^3 - 3x^2 + 2x} dx}
	\end{align*} 
}

%2.14
\subsection{
	\begin{align*}
		\int{\frac{x^2 - 3}{x^3 + x} dx}
	\end{align*} 
}

%2.15
\subsection{
	\begin{align*}
		\int{\frac{3x^2 - 6x + 9}{(x^2 + 9)(x - 3)} dx}
	\end{align*}
}

\newpage
\section{Quiz 4 Review}

%3.1
\subsection{Let $p(x) = \frac{x^2}{9}$ for $0 \leq x \leq 3$.}

%3.1.1
\subsubsection{Is $p(x)$ a valid probability density function?}

%.3.1.2
\subsubsection{Find $P(1 \leq X \leq 2)$.}

%3.1.3
\subsubsection{Find the mean $\mu$.}

%3.2
\subsection{Consider a random variable $X$ with probability density function $p(x) = k\sqrt{x + 10}$ on the interval $-6 \leq x \leq 6$.}

%3.2.1
\subsubsection{For what value(s) of $k$ is $p(x)$ a valid probability density function?}

%3.2.2
\subsubsection{Compute $P(X \leq 0)$.}

%3.2.3
\subsubsection{Find the mean $\mu$.}

%3.3
\subsection{For any positive value $\beta$ the function $p(x) = (\beta + 1)(\beta + 2)x^{\beta}(1 - x)$ is a probability density function on the interval $0 \leq x \leq 1$. Find the mean in terms of $\beta$.}

%3.4
\subsection{Find the arc length of the curve $y = \ln{\cos{x}}$ over the interval $[0, \frac{\pi}{4}]$.}

%3.5
\subsection{Find the arc length of the curve $y = \frac{1}{4}x^{3/2}$ over the interval $[0, 4]$.}

%3.6
\subsection{Find the arc length of the curve $y = \frac{1}{3}x^3 + \frac{1}{4x}$ over the interval $[1, 3]$.}

%3.7
\subsection{Find the surface area generated by rotating $y = 7x$ from $x = 0$ and $x = 1$ about the $x$-axis.}

%3.8
\subsection{Find the surface area generated by rotating $y = \sqrt{x}$ from $x = 0$ and $x = 2$ about the $x$-axis.}

%3.9
\subsection{Find the surface area generated by rotating $y = \sqrt{x} - \frac{1}{3}x^{3/2}$ from $x = 1$ and $x = 3$ about the $x$-axis.}

%3.10
\subsection{Find the surface area generated by rotating $y = x^3$ from $x = 0$ and $x = 1$ about the $x$-axis.}

%3.11
\subsection{Find the surface area generated by rotating $y = \sqrt{4 - x^2}$ from $x = -1$ and $x = 1$ about the $x$-axis.}









\newpage
\section{Quiz 5 Review}
%4.1
\subsection{The vertical wall on the end of a swimming pool is 20 ft wide and 8 ft high. If water in the swimming pool is filled to a height of 7 ft, find the force exerted on the wall by the water $(\rho = 62.4 $lb/ft$^3)$.}

%4.2
\subsection{The vertical gate of a damn has the shape of a trapezoid, 12 ft at the top, 8 ft at the bottom and 4 ft high. What is the force on the gate when the surface of the water is 2 ft above the top of the gate?}

%4.3
\subsection{The viewing port of a submarine is a circle of radius 1 ft. If the center of the viewing port is 100 ft below the surface, find the force exerted by the water on it.}

%4.4
\subsection{Find the force on a vertical flat plate in the form of a semicircle 5 meters in radius that is submerged in water $(\rho = 9810 $N/m$^3)$.}

Find the center of mass for:
%4.5
\subsection{The region bounded by $f(x) = x^2$ and the $x$-axis for $[0, 2]$.}

%4.6
\subsection{The region bounded by $f(x) = \sqrt{x}$ and the $x$-axis for $[0, 4]$.}

%4.7
\subsection{The region bounded by $f(x) = x^2 - 3$ and $g(x) = -x^2 + 2x + 1$.}

%4.8
\subsection{The region bounded by $f(x) = x^2 - 3$ and $g(x) = -x^2 + 2x + 1$.}

%4.9
\subsection{The region bounded by $\frac{x}{a} +\frac{y}{b} = 1$, the $x$-axis, and the $y$-axis.}

%4.10
\subsection{Verify that $y = \frac{x^4}{16}$ is a solution of the differential equation
	\begin{align*}
		\frac{dy}{dx} = xy^{1/2}
	\end{align*}
}

%4.11
\subsection{Verify that $y = x^2 + 2x + 2 + Ce^x$ is a solution of the differential equation
	\begin{align*}
		y' - y + x^2 = 0
	\end{align*}
}

Find the general solution of:
%4.12
\subsection{
	\begin{align*}
		\frac{dy}{dx} = (x + 1)^2
	\end{align*}
}

%4.13
\subsection{
	\begin{align*}
		y^2 y' = 3x^2
	\end{align*}
}

%4.14
\subsection{
	\begin{align*}
		y' = x^3y^2 + y^2
	\end{align*}
}

%4.15
\subsection{
	\begin{align*}
		y' = 5 - 2y
	\end{align*}
}

%4.16
\subsection{
	\begin{align*}
		\frac{dy}{dx} = \frac{x}{y^2}
	\end{align*}
}

%4.17
\subsection{
	\begin{align*}
		\frac{dy}{dx} = \frac{7}{y}
	\end{align*}
}

%4.18
\subsection{
	\begin{align*}
		x(y - 1)y' = y
	\end{align*}
}

%4.19
\subsection{Solve $\frac{dy}{dx} = 1 + y \quad y(0) = 5$.}

%4.20
\subsection{Solve $\frac{dy}{dx} = y^{2/3} \quad y(0) = 8$.}

%4.21
\subsection{Solve $\frac{dy}{dx} = y \quad y(0) = 3$.}








\newpage
\section{Quiz 6 Review}

%5.1
\subsection{A pie is taken out of the oven at a temperature of $200$F and put in a room with a temperature of $70$F. The temperature of the pie is $160$F after $15$ min.}
%5.1.1
\subsubsection{What is the temperature of the pie after $30$ min?}
%5.1.2
\subsubsection{When will the pie be $120$F?}

%5.2
\subsection{Henry hides his bagel in a refrigerator (temperature $32$F). After $10$ minutes, the bagel's temperature is $80$F and after $20$ minutes it is $44$F. What is the temperature of the bagel when it was first put in the refrigerator?}

%5.3
\subsection{Sketch the slope field for $\frac{dy}{dx} = x(6 - y)$ and draw the solution that goes through $(0, 0)$.}

%5.4
\subsection{Sketch the slope field for $\frac{dy}{dx} = xy$ and draw the solution that goes through $(0, 1)$.}

Use Euler's method:
%5.5
\subsection{
	\begin{align*}
		\frac{dy}{dx} = y \quad y(0) = 1 \text{ find } y(1) \text{ with } h = 0.1
	\end{align*}
}

%5.6
\subsection{
	\begin{align*}
		\frac{dy}{dx} = 2y - 1 \quad y(0) = 1 \text{ find } y(1) \text{ with } h = 0.1
	\end{align*}
}

%5.7
\subsection{
	\begin{align*}
		\text{Solve } y' = 1.5y\bigg( 1 - \frac{y}{4} \bigg) \quad y(0) = 1
	\end{align*}
}

%5.8
\subsection{In one of the dorms there are $1000$ students. After fall break, $20$ students return with the flu and $5$ days later, $35$ students have the flu. If the number of students with the flu follows the logistic model, how many students will have the flu after $2$ weeks ($14$ days)?}

%5.9
\subsection{A fish farm is stocked with $100$ fish. Suppose that the fish population satisfies the logistic equation and that the carrying capacity of the pond is $2000$. If after $1$ year the population has increased to $250$,}
%5.9.1
\subsubsection{find an equation for the number of fish after $t$ years.}
%5.9.2
\subsubsection{How long will it take for the fish population to reach $1000$?}





\newpage
\section{Quiz 7 Review}
Determine if the following sequences converge or diverge. If it converges, find the limit.
%6.1
\subsection{
	\begin{align*}
    		\biggl\{ \frac{2n - 1}{3n^2 + 1} \biggl\}_{n = 1}^{\infty}
	\end{align*}
}

%6.2
\subsection{
	\begin{align*}
    		\biggl\{ \frac{-9 + (-1)^n}{n!} \biggl\}_{n = 1}^{\infty}
	\end{align*}
}

%6.3
\subsection{
	\begin{align*}
    		\biggl\{ \bigg( \frac{n +1}{n} \bigg)^n \biggl\}_{n = 1}^{\infty}
	\end{align*}
}

%6.4
\subsection{
	\begin{align*}
    		\biggl\{ \frac{2n - \sqrt{n}}{n} \biggl\}_{n = 1}^{\infty}
	\end{align*}
}

Determine if the following sequences are increasing or decreasing.
%6.4
\subsection{
	\begin{align*}
    		\biggl\{ 7 - \frac{1}{n^2} \biggl\}_{n = 1}^{\infty}
	\end{align*}
}

%6.6
\subsection{
	\begin{align*}
    		\biggl\{ \frac{2^n}{n!} \biggl\}_{n = 4}^{\infty}
	\end{align*}
}

%6.7
\subsection{
	\begin{align*}
    		\{n e^{-n} \}_{n = 1}^{\infty}
	\end{align*}
}

%6.8
\subsection{
	\begin{align*}
    		\{ 12 \sin{(3n)} \}_{n = 1}^{\infty}
	\end{align*}
}

%6.9
\subsection{Use the fact that $\sum_{n = 1}^{\infty} \frac{1}{n^2} = \frac{\pi^2}{6}$ to find $\sum_{n = 3}^{\infty} \frac{1}{n^2}$.}

Determine if the following series converge or diverge, and if it converges, find the sum:
%6.10
\subsection{
	\begin{align*}
		\sum_{n = 0}^{\infty} \frac{1}{3^n}
	\end{align*}
}

%6.11
\subsection{
	\begin{align*}
		\sum_{n = 1}^{\infty} 2(-0.9)^n
	\end{align*}
}

%6.12
\subsection{
	\begin{align*}
		\sum_{n = 1}^{\infty} \frac{n - 6}{n}
	\end{align*}
}

%6.13
\subsection{
	\begin{align*}
		\sum_{n = 1}^{\infty} \frac{1}{1 + e^{-n}}
	\end{align*}
}

%6.14
\subsection{
	\begin{align*}
		\sum_{n = 0}^{\infty} 0.5 \bigg( -\frac{4}{3} \bigg)^n
	\end{align*}
}

%6.15
\subsection{
	\begin{align*}
		\sum_{n = 0}^{\infty} \bigg( \frac{\sqrt{5}}{1 + \sqrt{5}} \bigg)^n
	\end{align*}
}

%6.16
\subsection{
	\begin{align*}
		\sum_{n = 1}^{\infty} \bigg( \frac{1}{n} - \frac{1}{n + 2} \bigg)
	\end{align*}
}

%6.17
\subsection{
	\begin{align*}
		\sum_{n = 1}^{\infty} \frac{1}{n^2 + 4n + 3}
	\end{align*}
}

%6.18
\subsection{
	\begin{align*}
		\sum_{n = 1}^{\infty} \frac{1}{n(n + 2)}
	\end{align*}
}








\newpage
\section{Quiz 8 Review}
Determine if the following series converge or diverge.
%7.1
\subsection{
	\begin{align*}
		\sum_{n = 2}^{\infty} \frac{(\ln{n})^2}{n}
	\end{align*}
}

%7.2
\subsection{
	\begin{align*}
		\sum_{n = 2}^{\infty} \frac{1}{n\sqrt{\ln{n}}}
	\end{align*}
}

%7.3
\subsection{
	\begin{align*}
		\sum_{n = 2}^{\infty} \frac{1}{n(\ln{n})^2}
	\end{align*}
}

%7.4
\subsection{
	\begin{align*}
		\sum_{n = 1}^{\infty} \frac{1}{n^{-2}}
	\end{align*}
}

%7.5
\subsection{
	\begin{align*}
		\sum_{n = 1}^{\infty} \frac{1}{2\sqrt{n} + \sqrt[3]{n}}
	\end{align*}
}

%7.6
\subsection{
	\begin{align*}
		\sum_{n = 1}^{\infty} \frac{3 + \cos{n}}{n^2}
	\end{align*}
}

%7.7
\subsection{
	\begin{align*}
		\sum_{n = 1}^{\infty} \frac{\sin^2{n}}{2^n}
	\end{align*}
}

%7.8
\subsection{
	\begin{align*}
		\sum_{n = 1}^{\infty} \bigg( \frac{n}{3n + 1} \bigg)^n
	\end{align*}
}

%7.9
\subsection{
	\begin{align*}
		\sum_{n = 1}^{\infty} \frac{n!}{10^n}
	\end{align*}
}

%7.10
\subsection{
	\begin{align*}
		\sum_{n = 1}^{\infty} \frac{n^{\sqrt{2}}}{2^n}
	\end{align*}
}

%7.11
\subsection{
	\begin{align*}
		\sum_{n = 1}^{\infty} \frac{1}{1 +\ln{n}}
	\end{align*}
}

%7.12
\subsection{
	\begin{align*}
		\sum_{n = 1}^{\infty} \frac{1}{\sqrt{n} \ln{n}}
	\end{align*}
}

%7.13
\subsection{
	\begin{align*}
		\sum_{n = 1}^{\infty} \frac{(\ln{n})^n}{n^n}
	\end{align*}
}

%7.14
\subsection{
	\begin{align*}
		\sum_{n = 1}^{\infty} \frac{\cos{(\frac{1}{n})}}{2n + 3}
	\end{align*}
}

%7.15
\subsection{
	\begin{align*}
		\sum_{n = 1}^{\infty} \frac{4 + |\cos{n}|}{n^3}
	\end{align*}
}

%7.16
\subsection{
	\begin{align*}
		\sum_{n = 1}^{\infty} \frac{(n + 4)!}{4!n!4^n}
	\end{align*}
}















\newpage
\section{Quiz 9 Review}
Determine if the following series converge absolutely, converge conditionally, or diverge:
%8.1
\subsection{
	\begin{align*}
		\sum_{n = 1}^{\infty} (-1)^n \frac{1}{\sqrt{n + 1}}
	\end{align*}
}

%8.2
\subsection{
	\begin{align*}
		\sum_{n = 1}^{\infty} (-1)^n \frac{n!}{2^n}
	\end{align*}
}


%8.3
\subsection{
	\begin{align*}
		\sum_{n = 1}^{\infty} (-1)^n \frac{\arctan{n}}{n^2 + 1}
	\end{align*}
}

%8.4
\subsection{
	\begin{align*}
		\sum_{n = 1}^{\infty} (-1)^n \bigg( \frac{\ln{n}}{\ln{n^2}} \bigg)^n
	\end{align*}
}

%8.5
\subsection{
	\begin{align*}
		\sum_{n = 1}^{\infty} (-1)^n \frac{n}{n + 1}
	\end{align*}
}

%8.6
\subsection{
	\begin{align*}
		\sum_{n = 1}^{\infty} (-1)^n \frac{\sin{n}}{n^2}
	\end{align*}
}

%8.7
\subsection{
	\begin{align*}
		\sum_{n = 1}^{\infty} (-1)^n \frac{1}{\sqrt{n +1} + \sqrt{n}}
	\end{align*}
}

%8.8
\subsection{
	\begin{align*}
		\sum_{n = 1}^{\infty} (-1)^n \frac{(2n)!}{2^n n! n}
	\end{align*}
}

%8.9
\subsection{
	\begin{align*}
		\sum_{n = 1}^{\infty} \frac{(-1)^{n + 1} n^3}{e^n}
	\end{align*}
}

%8.10
\subsection{
	\begin{align*}
		\sum_{n = 1}^{\infty} \frac{(-1)^{n} \ln{n}}{n}
	\end{align*}
}

Find the radius and interval of convergence for:
%8.11
\subsection{
	\begin{align*}
		\sum_{n = 1}^{\infty} 3^n x^n
	\end{align*}
}

%8.12
\subsection{
	\begin{align*}
		\sum_{n = 1}^{\infty} \frac{n!}{3^n} (x - 2)^n
	\end{align*}
}

%8.13
\subsection{
	\begin{align*}
		\sum_{n = 1}^{\infty} \frac{(x + 2)^n}{n + 1}
	\end{align*}
}

%8.14
\subsection{
	\begin{align*}
		\sum_{n = 1}^{\infty} \frac{1}{n^2 + 1} (x - 3)^n
	\end{align*}
}

%8.15
\subsection{
	\begin{align*}
		\sum_{n = 1}^{\infty} \frac{9^n}{n!} (x - 1)^n
	\end{align*}
}

Find the $n$-th degree Taylor polynomial centered at $x = a$ for:
%8.16
\subsection{
	\begin{align*}
		f(x) = \ln{x} \quad a = 1 \quad n = 4
	\end{align*}
}

%8.17
\subsection{
	\begin{align*}
		f(x) = \arctan{x} \quad a = 0 \quad n = 2
	\end{align*}
}

%8.18
\subsection{
	\begin{align*}
		f(x) = \cos{x} \quad a = \frac{\pi}{4} \quad n = 3
	\end{align*}
}

%8.19
\subsection{
	\begin{align*}
		f(x) = \frac{1}{\sqrt{1 - x}} \quad a = 0 \quad n = 4
	\end{align*}
}







\newpage
\section{Quiz 10 Review}
Write out the first four non-zero terms for the Maclaurin series for:
%9.1
\subsection{
	\begin{align*}
		f(x) = \sin{2x}
	\end{align*}
}

%9.2
\subsection{
	\begin{align*}
		f(x) = x\sin{x}
	\end{align*}
}

%9.3
\subsection{
	\begin{align*}
		f(x) = xe^x
	\end{align*}
}

%9.4
\subsection{
	\begin{align*}
		f(x) = e^x \sin{x}
	\end{align*}
}

%9.5
\subsection{
	\begin{align*}
		f(x) = \ln{\bigg[ \frac{1 + x}{1 - x} \bigg]}
	\end{align*}
}

%9.6
\subsection{For the parametric curve: $x = 4t + 6, y = 8t + 2$, eliminate the parameter and express in rectangular form.}

%9.7
\subsection{For the parametric curve: $x = 2 + 2\cos{t}, y = 3 + 4\sin{t}$, eliminate the parameter and express in rectangular form.}

%9.8
\subsection{Find the equation of the tangent line to the curve $x = 2t^2 + 2t, y = 2t^2 + 4t$ at $t = 1$.}

%9.9
\subsection{Find the equation of the tangent line to the curve $x = \sec{t}, y = \tan{t}$ at $t = \frac{\pi}{4}$.}

%9.10
\subsection{Find $\frac{d^2y}{dx^2}$ if $x = \sin{t}, y = 2\cos{t}$.}

%9.11
\subsection{Find the arc length of $x = 4\sin{2t}, y = 4\cos{2t}$, over the interval $0 \leq t \leq \frac{\pi}{2}$.}

%9.12
\subsection{Find the arc length of $x = t^2, y = t^3$, from $(1, 1)$ to $(4, 8$.}

%9.13
\subsection{Find the arc length of $x = a(t - \sin{t}), y = a(1 - \cos{t})$, for $0 \leq t \leq 2\pi$.}









\newpage
\section{Quiz 11 Review}
%10.1
\subsection{A vector $\vec{v}$ has initial point $(-4, -1)$ and terminal point $(-2, -5)$/ Find $\vec{v}$.}

%10.2
\subsection{A vector $\vec{v}$ has initial point $(1, 8)$ and terminal point $(3, -7)$/ Find its magnitude.}

%10.3
\subsection{Given $\vec{w} = \hat{i}$ and $\vec{u} = 4\hat{i} - 2\hat{j}$, find $\vec{v} = 3\vec{w} + 5\vec{u}$.}

%10.4
\subsection{Determine which vector is parallel to the vector $\vec{v} = <2, -3, -1>$
	\begin{align*}
		<4, 6, -2> \quad <\frac{-2}{3}, 1, \frac{1}{3}> \quad <1, \frac{-3}{2}, \frac{1}{2}>
	\end{align*}
	\begin{align*}
		<6, -9, 3> \quad \text{None of these}
	\end{align*}
}

%10.5
\subsection{Find the center and radius of the sphere given by:
	\begin{align*}
		x^2 + y^2 + z^2 + 2x - 2y + 6z + 7 = 0
	\end{align*}
}

%10.6
\subsection{Find the center and radius of the sphere given by:
	\begin{align*}
		x^2 + y^2 + z^2 -4x + 8y - 10z + 20 = 0
	\end{align*}
}

%10.7
\subsection{Is the origin inside or outside the sphere
	\begin{align*}
		x^2 - 2x + y^2 + 4y + z^2 - 6z = 2
	\end{align*}
}

%10.8
\subsection{Find the parametric equations of the line through the points $(2, 0, 3)$ and $(4, 3, 3)$.}

%10.9
\subsection{Find the parametric equations of the line through the points $(-3, 2, 0)$ and $(4, 3, 3)$.}

%10.10
\subsection{Find numbers $x$ and $y$ such that the point $(x, y, 1)$ lies on the line passing through the points $(2, 5, 7)$ and $(0, 3, 2)$.}

%10.11
\subsection{Find the point where the line through $(3, 2, 4)$ with direction vector $\vec{v} = <7, 5, -4>$ intersects the $xy$-place.}

%10.12
\subsection{Determine whether the lines intersect, are parallel, or skew:
	\begin{align*}
		L_1: x = 4t + 2 \quad y = 3 \quad z = -t + 1
	\end{align*}
	\begin{align*}
		L_2: x = 2s + 2 \quad y = 2s + 3 \quad z = s + 1
	\end{align*}
}


%10.13
\subsection{Determine whether the lines intersect, are parallel, or skew:
	\begin{align*}
		L_1: x = 1 - 4t \quad y = 2 + 3t \quad z = 4 - 2t
	\end{align*}
	\begin{align*}
		L_2: x = 2 - s \quad y = 1 + s \quad z = 2 + 6s
	\end{align*}
}

%10.14
\subsection{For vectors $\vec{a} = <1, 2, -1>$ and $\vec{b} = <3, 5, -1>$ find:}
%10.14.1
\subsubsection{Find $2\vec{a} - 5\vec{b}$}
%10.14.2
\subsubsection{$\vec{a} \cdot \vec{b}$}
%10.14.3
\subsubsection{The $\cos{\theta}$ where $\theta$ is the angel between $\vec{a}$ and $\vec{b}$.}
%10.14.4
\subsubsection{A unit vector in the direction of $\vec{a}$.}
%10.14.5
\subsubsection{A vector of length 3 in the direction of $\vec{a}$.}

%10.15
\subsection{For vectors $\vec{a} = <1, 1, -2>$ and $\vec{b} = <1, 2, -1>$ find:}
%10.15.1
\subsubsection{Find $4\vec{a} - 3\vec{b}$}
%10.15.2
\subsubsection{$\vec{a} \cdot \vec{b}$}
%10.15.3
\subsubsection{The $\cos{\theta}$ where $\theta$ is the angel between $\vec{a}$ and $\vec{b}$.}
%10.15.4
\subsubsection{A unit vector in the direction of $\vec{a}$.}
%10.15.5
\subsubsection{A vector of length 4 in the direction of $\vec{a}$.}

%10.16
\subsection{Find the value of $x$ so that $\vec{c} = <2, x, -3>$ and $\vec{d} = <-1, 3, -2>$ are perpendicular.}

%10.17
\subsection{Let $\vec{a} = <3, -1, 2>$ and $\vec{b} = <-2, -3, 2>$. Calculate $proj_{\vec{b}} \vec{a}$}

%10.18
\subsection{Let $\vec{a} = <3, -1, 2>$ and $\vec{b} = <-2, -3, 2>$. Calculate $proj_{\vec{a}} \vec{b}$}

















\end{document}
